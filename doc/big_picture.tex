\documentclass[12pt,a4paper]{article}
\usepackage[utf8x]{inputenc}
\usepackage[T1]{fontenc}
\usepackage{graphicx}
\usepackage{ucs}
\usepackage[polish]{babel}
\usepackage{geometry}

\newenvironment{short_list}{
  \begin{itemize}
  \setlength{\itemsep}{1pt}
  \setlength{\parskip}{0pt}
  \setlength{\parsep}{0pt}
}{
  \end{itemize}
}

\title{ZPR\\
Dokumentacja wstępna }

\author{
Paweł Szewczyk
\and
Konrad Ziaja
}


\geometry{a4paper}

\begin{document}
\maketitle

\section{Wstęp}
Celem projektu jest implementacja serwera wieloosobowej gry w pokera oraz prostego klienta gry.

\section{Założenia projektowe}
Podstawową funkcjonalnością projektu jest pełna obsługa wieloosobowej gry w pokera graczy pracujących na różnych platformach. Gra będzie udostępniała różne warianty 
pokera z opcją prostego dodania nowych wariantów.

\subsection{Dodatkowe funkcje dostępne dla użytkowników}
Możliwość przeprowadzenia partii między wieloma graczami rozszerzona zostanie o dodatkowe funkcje. Użytkownicy zyskają możliwość rejestracji konta w grze i śledzenia postępów swoich i swoich znajomych w rankingu.

\subsection{Możliwość rozszerzenia funkcjonalności}
Mechanika gry zostanie zdefiniowana poprzez skrypty po stronie serwera (Python), udostępniając interfejs umożliwiający rozszerzanie jej o nowe funkcje w sposób nie ingerujący w silnik gry. Tworzenie i ładowanie nowych modułów powinno być stosunkowo proste dla użytkowników.

\section{Wstępne szczegóły implementacyjne}
\subsection{Wybór narzędzi}
Projekt zostanie zrealizowany przy użyciu języka C++ z biblioteką Boost oraz skryptów w Pythonie. Serwer będzie również korzystał z bazy danych PostgreSQL.

\subsection{Architektura systemu}
Ogólną budowę projektowanego systemu ilustruje rysunek.\\
\includegraphics[scale=1]{architecture-overview.jpg}
\end{document}